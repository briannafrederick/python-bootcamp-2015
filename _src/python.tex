\chapter{Python review}

\section{Introduction}

\subsection{Interpreter}
\subsection{Variables}
\subsection{Modules and files}

References
\begin{itemize}
\item \url{https://docs.python.org}
\item \url{https://scipy-lectures.github.io/}
\end{itemize}


\section{Data Structures}

\subsection{Numbers}

\begin{minted}{python}
In [1]: string1 = "my string"

In [2]: string1.upper()
Out[2]: 'MY STRING'
\end{minted}

\subsection{Strings}
\subsection{Tuples}
\subsection{List}
\subsection{Dictionaries}
\subsection{Sets}

\section{Control flow}

\subsection{If-then-else}
\subsection{For-loops}
\subsection{While-loops}

\section{Advanced topics}

\subsection{Comprehension}
\subsection{Functions}
\subsection{Context managers}


\section{Data formats}

\subsection{CSV}

\subsection{JSON}

\subsection{HTML}


\begin{verbatim}
$ git clone https://github.com/dlab-berkeley/python-fundamentals.git
$ cd python-fundamentals/cheat-sheets/
$ ipython notebook Web-Scraping.ipynb
\end{verbatim}


%\href{https://github.com/dlab-berkeley/python-fundamentals/blob/master/cheat-sheets/Web-Scraping.ipynb}{IPython notebook}
%(\href{http://nbviewer.ipython.org/github/dlab-berkeley/python-fundamentals/blob/master/cheat-sheets/Web-Scraping.ipynb}{Rendered version})

\section{Standard library}

\subsection{re}

\section{Exercise}



