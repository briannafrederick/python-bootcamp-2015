\chapter{NumPy, SciPy, and matplolib}

\begin{abstract}
Explore Python's core numerical, scientific, and plotting packages. For background
see:
\begin{itemize}
\item Fernando Pérez, Brian E. Granger, and John D. Hunter. "Python: an ecosystem for
scientific computing." \emph{Computing in Science \& Engineering} 13, no. 2 (2011):
13-21.
\item Stéfan van der Walt, S. Chris Colbert, and Gael Varoquaux. "The NumPy array: a
structure for efficient numerical computation." \emph{Computing in Science \&
Engineering} 13, no. 2 (2011): 22-30.
\item John D. Hunter. "Matplotlib: A 2D graphics environment." \emph{Computing
in Science \& Engineering} 9, no. 3 (2007): 0090-95.
\end{itemize}
\end{abstract}

\section{Introduction}
\begin{itemize}
\item \url{https://scipy-lectures.github.io/}
\item \url{http://docs.scipy.org/doc/}
\end{itemize}

\section{NumPy and matplotlib}

\subsection{ndarray: the fundamental datastructure for scientific computing}
\subsection{indexing, fancy indexing, and slicing}
\subsection{2D plotting}
\subsection{basic operations}

\section{Example: random walk redux}

%% begin.rcode p1, engine='python'
%% end.rcode

%% begin.rcode p2, engine='python'
%% end.rcode

\subsection{Exercise}
Write a Python script to construct Sierpinski Gasket using the following
algorithm:

\begin{enumerate}
\item Choose 3 points in the plane (forming a triangle).
\item Choose another "starting" pointing (current position).
\item Randomly choose one of the corners of the triangle.
\item Move halfway from your current position to the selected corner.
\item Plot the new current position.
\item Repeat from step 3 (for 100 times).
\end{enumerate}

\section{SciPy}

\subsection{scipy.io}
\subsection{scipy.signal}
\subsection{scipy.stats}
\subsection{scipy.linalg}
\subsection{scipy.optimize}

\section{Exercise}
